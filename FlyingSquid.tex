\documentclass {article}
\usepackage{amsmath}
\usepackage{appendix}
\usepackage{hyperref}
\hypersetup{
    colorlinks=true, % make the links colored
    linkcolor=black, % color TOC links in black
    urlcolor=red, % color URLs in red
    linktoc=all % 'all' will create links for everything in the TOC
}
\usepackage{float}
\usepackage{graphicx}
\graphicspath{{images/}}
\usepackage{csquotes}
\usepackage{fancyvrb}

\usepackage[backend=bibtex,style=numeric]{biblatex}
\addbibresource{sources.bib}


\usepackage{listings}
\lstset{
basicstyle=\small\ttfamily,
columns=flexible,
frame=single,
breaklines=true,
language=C++,
keywordstyle=\color{magenta}\ttfamily,
stringstyle=\color{red}\ttfamily,
commentstyle=\color{blue}\ttfamily,
}

\usepackage{tikz}
\usetikzlibrary{arrows}

\title {
	{\textbf{Flying Squid}\\}
	{Tufts University: Senior Design Project\\}
	{\large Advisor: Prof. Fahad Dogar}
}

\author {Sunjay Bhatia, Victor Chao, Jim Mao, and Siddhartha Prasad}
\date{Spring 2016}

\begin {document}
\maketitle

\noindent
Flying Squid is a proxy that can provide a distributed, cloud based cache localized at the edge of a network. The system can eliminate unnecessary bandwidth use, and provide faster data delivery than content providers by themselves. Flying Squid aims to be a truly web-based personal proxy, with shared cloud based caches and value based partial caching. By improving content delivery speed and reducing client data usage, the proxy will be especially useful to mobile users and those with spotty network connectivity.

\tableofcontents
\pagebreak



\section{Background}

Proxy servers help communication between two entities on a network. They often act as intermediaries between clients and content-delivering servers. They help optimize and add structure to networks and distributed systems. The latency incurred from downloading content directly from target servers is fast becoming a limiting factor on ‘internet speeds’. With sharp upward trends in the number of devices connected to the internet, this has forced cell-phone providers, ISPs and even large local area networks to use these servers to optimize both networks and content delivery. As a result, end users consciously and unconsciously end up using several proxies a day. While these proxies have sophisticated caching and compression mechanisms, they are stand-alone programs that do not do the most they can to share information about users and the data they are caching.

\begin{figure}[H] \centering
\includegraphics[width=10cm]{TraditionalArch}
\caption{Traditional Client-Proxy Architecture}
\end{figure}


In Figure 1 we can see a simple representation of the traditional role proxies have. Users connect to proxy servers as intermediaries to their connection to a network, such as the internet. Each of these proxies traditionally has its own disk and RAM cache. It is also possible to utilize these proxies as a distributed network with a shared cache represented by \lq joining \rq each of the disk caches into a unified cache. However, this cache still only incorporates the basic levels of disk and RAM storage and there is potentially significant overhead in communicating between proxies to locate content that may be on a different physical machine than the one handling the request and serving the content.



\section{Flying Squid}

\subsection{Apache Traffic Server}

Apache Traffic Server (ATS) is a high-performance open-source proxy that was built by Inktomi and Yahoo!. It is modular in nature, and is known for its efficient caching mechanisms. According to the ATS documentation, it \lq \lq is designed to improve content delivery for enterprises, Internet service providers (ISPs), backbone providers, and large intranets by maximizing existing and available bandwidth \rq \rq \cite{ATS}.

Flying Squid is an Apache Traffic Server augmentation that can provide a distributed, cloud based cache localized at the edge of the network. The system utilizes custom caching techniques and protocols to reduce uneccesary bandwidth use. Thus, it provides faster data delivery than content providers by themselves. 


Flying Squid aims to be a truly web-based personal proxy, with shared cloud based caches and value-based storage and transfer protocols. By improving content delivery speed and reducing client data usage, the proxy is especially useful to mobile users and those with spotty network connectivity.

\section{Cloud Caching}

FlyingSquid leverages different cloud storage tiers, each with unique benefits, to build a more sophisticated caching infrastructure. As a result, individual instances in an ATS cluster can be configured to share access to cached objects and use storage space more efficiently.

\begin{figure}[H] \centering
\includegraphics[width=\textwidth]{CloudCachingArch}
\caption{Flying Squid Proxy Cloud Caching Architecture}
\end{figure}

\subsection{Augmented Tier - Redis Cluster}

\subsection{Storage Tier - AWS Simple Storage Service}

\subsection{Storage Tier - AWS Cloudfront}


\section{Fingerprinting: Value Based Caching}

 Whether an application uses compression or intricate caching mechanisms, the easiest way to improve content delivery speed is to \textit{send less over the network}. With this in mind, a lot of research has been done with respect to protocols focused on eliminating redundant data transfers over HTTP links. Much of this redundancy is caused by the first-class nature of the \lq file \rq in traditional caching systems.\\

\noindent
Inspired by research at Berkley on value based caching\cite{Rhea09}, Flying Squid's Fingerprinting system caches with respect to data, not file name. As a result, only changed and new parts of content need ever retransmitted over a network. 


\subsection{The API}

Flying Squid's value based caching protocol takes the form of a simple, C++ class-based library. Modules exist on both the client and user side.

\subsubsection{Client}

\begin{lstlisting}
class RabinClient
{
    public:
        /* Server name and port number*/
        RabinClient(char * hostname, int port_);
       
	 ~RabinClient();

        /*
        * Receives a file from the server into the open, write enabled file pointer 'file'.
        * Returns the number of blocks added to the file.
	 * This is a blocking call.
        */
        unsigned receive_file(FILE *file); 

       /* Establishes a connection to the server */
        int connect_to_server();

	 /* Disconnects from the server */
        int disconnect_from_server();

};

\end{lstlisting}

\subsubsection{Server}

\begin{lstlisting}
class RabinServer
{

    public:
        /* Port at which to be open*/
        RabinServer(int port_);
        ~RabinServer();

        /* Sends a file of size s to the client.
		This is a blocking call.
	*/
        int send_file(char *file, size_t s); 


        /* Listens for the client and accepts a connection 
        * Must be called before write_to_client is called  */
        int connect_to_client();

	 /* Connects to the client */
        int disconnect_from_client();

};

\end{lstlisting}

\subsection{Example: Integration}

\begin{figure}[H] \centering
\includegraphics[width=\textwidth]{FingerprintingArch}
\caption{Flying Squid Value Based Caching Architecture}
\end{figure}

Flying Squid uses a modified TCP forwarding servers \cite{Partow} to interface with ATS. When the user makes a request through the browser, the browser forwards the request through its interfacing layer to the interfacing layer of ATS. ATS receives this request and gathers the requested data file from the Internet and the cloud caching mechanism previously described. ATS then sends over the file to the Rabin Server, which breaks down the file into blocks. The Rabin Server sends each block to the Rabin Client, which reassembles the blocks into a file. The Rabin Client then sends the response to the browser through the browser's interfacing layer. Lastly, the browser displays the response to the user. 

\subsection{Technical Approach}

Flying Squid's Rabin Fingerprinting library provides a clean file transfer abstraction to the user, hiding the intricacies of a partial file transfer abstraction.

\begin{enumerate}

\item Server-side files are split into blocks of maximum size $1 KB$ using a \verb|djb-2|\cite{djb2} based uniform random function of the form: \\
$f: \{\text{byte} * \text{byte} * \text{byte} \} \rightarrow \{0, 1, ..., 1023\}$
 
\item Block boundaries are determined when this function returns 0. These blocks are treated as first-class objects and are hashed locally. A custom protocol can then be used to transfer the blocks representing a file over a TCP connection. Only the hash digests (identifiers) of previously transmitted blocks are transferred. Blocks of size 0 are used to denote $EOF$.

\item The client  leverages the ordering properties of TCP to receive blocks in order. These are then locally stored, and identifiers are used to re-assemble them into traditional files. 
\end{enumerate}

\subsection {Statistics}

Pairs of files were generated that differ in the following ways:

\begin{enumerate}
\item Haskell: One file has an added byte.
\item HTML: Files differ by one large contiguous chunk.
\item C: Files differ by a large number of chunks distributed through the file
\end{enumerate}

For each pair, these pairs of files were sent one after the other over a network using the Flying Squid fingerprinting system, with different maximum block sizes. Max block sizes 	smaller than $256$ bytes were found to cause significantly more conflicts with the internal \verb|djb2| hash. 


\begin{figure}[H] \centering
\includegraphics[width=\textwidth]{HaskellGraph.png}
\caption{Details the percent of the second file sent at different max block sizes.}
\end{figure}
\noindent
For files differing by one byte, a balance has to be struck between several small block headers being sent and the size of the retransmitted block(s). At most one block will be retransmitted due to the small difference between files. An optimal block size was found around $1000$ bytes

\begin{figure}[H] \centering
\includegraphics[width=\textwidth]{HTMLGraph.png}
\caption{Details the percent of the second file sent at different max block sizes.}
\end{figure}
\noindent
For files differing by a large contiguous chunk a minimum was found around $1400$ bytes.

\begin{figure}[H] \centering
\includegraphics[width=\textwidth]{CGraph.png}
\caption{Details the percent of the second file sent at different max block sizes.}
\end{figure}
\noindent
For files differing by a large number of chunks distributed through the file, smaller max block sizes were generally found to be better.


\begin{figure}[H] \centering
\includegraphics[width=\textwidth]{AllGraph.png}
\caption{Details the percent of the second file sent at different max block sizes.}
\end{figure}
\noindent
Taking the average of all these files, it was found that a minimal amount was resent around a maximum block size of $1 KB$. As a result, the maximum value chosen was $1024$ bytes, with an expected max size of $512$ bytes.



\section{Open Source Component}

\noindent
We have been working with an open source fork of ATS hosted in our GitHub organization. The modifications we have made to ATS will hopefully be proposed as pull requests we can make to the main project. Several of our contributions have been placed directly in the ATS codebase. Thus, all modifications are open source contributions that we hope to be approved as part of the main Traffic Server codebase. \\

\noindent
The \lq Rabin \rq fingerprinting component of the project is also open source. It functions is a standalone module that can be easily included in C++ code bases.\\

\noindent
All of our GitHub repositories can be found at \url{https://github.com/FlyingSquid/}.

\section{Issues and Challenges}


\noindent
The ATS project dates back to 2009, when Yahoo! released the source code to the Apache software foundation. The result is a large code base on GitHub that gets updated constantly. The biggest challenge we faced throughout the project was understanding the code base, in which project components invoke and pass data to one another by registering event handlers. In addition, the session nature of operations in ATS means that data are not always readily available and do not persist once a session expires. \\

\noindent
The cloud caching module of FlyingSquid was fully integrated into the ATS code base. Because the caching and the HTTP request/response handling modules are well-encapsulated in their respective classes, and the cloud caching module requires data across the abstraction barrier, we had to implement parts of the FlyingSquid in the HTTP module and others in the caching module.\\ 

\section{Further Work}

\noindent
A lot of work needs to be done to make FlyingSquid a production level proxy. The proxy is only as strong as its weakest link. The integration of cloud caches, ATS and fingerprinting systems could be further streamlined. Better integration would involve more robust browser-side clients, and the elimination of redundant actions within ATS. Value based caching could be made more flexible, with adjustable block sizes and less persistent TCP connections. More benchmark and bandwidth analysis could help better understand the strengths and weaknesses of Flying Squid. Eventually, this analysis would provide grounds for interesting research.\\

\noindent
To accurately measure the effectiveness and relevance of Flying Squid, we must investigate several use cases for our proxy server. We must leverage these use cases to show, with measurable benchmarks, that Flying Squid is indeed an improvement on the proxy status quo. First, there is the use case of a proxy network with multiple nodes to maximize personalized caching for more than just an individual. The custom caching could cater to a corporation, non-profit organization, as well as universities like Tufts. 
Many countries in the world have extremely limited bandwidth. We must account for these use cases, as these countries’ population constitute a significant part of the world’s internet users. In effect, Flying Squid will deliver content to these users at speeds faster than the speed that is currently available.









\pagebreak

\begin{appendices}

\section{User Manual}

\subsection{ATS with Cloud Caching}

\subsection{Value-Based Caching Integration}

\section{Better Integration: Google Native Client}

\paragraph{Idea}

The idea behind using Google Native Client is simply to eliminate the browser's interfacing layer. With Google Native Client, the browser can run C++ code in the browser. This way, the request does not have to pass through the TCP proxy interfacing layer and can go straight to ATS. Not only would this make Flying Squid more secure and portable, but it would also bring huge performance benefits.

Javascript, HTML, CSS

\paragraph{Setting up Native Client}

Before Native Client can be set up, Python 2.7 and Make need to be available on the machine. Once these executables are available, the Native Client executable can be downloaded at the following website: https://developer.chrome.com/native-client/sdk/download. Then perform the following steps:

\begin{enumerate} 
\item Unzip the SDK zip. On Mac/Linux the command is:

\begin{lstlisting} [language=bash] 
unzip nacl_sdk.zip
\end{lstlisting}

On Windows, right-click on the .zip file and click \lq Extract All... \rq

\item  Update the Native Client bundles. On Mac/Linux, the command is:

\begin{lstlisting} [language=bash] 
./naclsdk update
\end{lstlisting}

On Windows, the command is:

\begin{lstlisting} [language=bash] 
naclsdk update
\end{lstlisting}

\item Start the local server. Perform the following commands:

\begin{lstlisting} [language=bash] 
cd pepper_${VERSION}/getting_started
make serve
\end{lstlisting}

\item Access the application at:

\begin{verbatim}
http://localhost:5103
\end{verbatim}

\end{enumerate}

That's all that's required to set up a barebones version of the Native Client. To get an example up and running, please see the Hello World tutorial at: https://developer.chrome.com/native-client/devguide/tutorial/tutorial-part1.

\paragraph{Challenges}

Unfortunately, there were several roadblocks that prevented Google Native Client from being a viable option for Flying squid at the moment. First and foremost, 

\end{appendices}

\printbibliography

\end{document}
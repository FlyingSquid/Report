\begin{appendices}

\section{User Manual}

\subsection{ATS with Cloud Caching}

\subsection{Value-Based Caching Integration}

\section{Better Integration: Google Native Client}

\paragraph{Idea}

The idea behind using Google Native Client is simply to eliminate the browser's interfacing layer. With Google Native Client, the browser can run C++ code in the browser. This way, the request does not have to pass through the TCP proxy interfacing layer and can go straight to ATS. Not only would this make Flying Squid more secure and portable, but it would also bring huge performance benefits.

Javascript, HTML, CSS

\paragraph{Setting up Native Client}

Before Native Client can be set up, Python 2.7 and Make need to be available on the machine. Once these executables are available, the Native Client executable can be downloaded at the following website: https://developer.chrome.com/native-client/sdk/download. Then perform the following steps:

\begin{enumerate} 
\item Unzip the SDK zip. On Mac/Linux the command is:

\begin{lstlisting} [language=bash] 
unzip nacl_sdk.zip
\end{lstlisting}

On Windows, right-click on the .zip file and click \lq Extract All... \rq

\item  Update the Native Client bundles. On Mac/Linux, the command is:

\begin{lstlisting} [language=bash] 
./naclsdk update
\end{lstlisting}

On Windows, the command is:

\begin{lstlisting} [language=bash] 
naclsdk update
\end{lstlisting}

\item Start the local server. Perform the following commands:

\begin{lstlisting} [language=bash] 
cd pepper_${VERSION}/getting_started
make serve
\end{lstlisting}

\item Access the application at:

\begin{verbatim}
http://localhost:5103
\end{verbatim}

\end{enumerate}

That's all that's required to set up a barebones version of the Native Client. To get an example up and running, please see the Hello World tutorial at: https://developer.chrome.com/native-client/devguide/tutorial/tutorial-part1.

\paragraph{Challenges}

Unfortunately, there were several roadblocks that prevented Google Native Client from being a viable option for Flying squid at the moment. First and foremost, 

\end{appendices}
\begin{appendices}

\section{User Manual}

\subsection{Apache Traffic Server}

To run barebones, non-augmented version of ATS, run the following commands:

\begin{enumerate}

\item Clone the FlyingSquid scripts repository

\begin{lstlisting} [language=bash] 
git clone https://github.com/FlyingSquid/scripts.git
\end{lstlisting}

\item Change into the setup directory and execute the setup script.

\begin{lstlisting} [language=bash] 
cd ./scripts/setup
./setup.sh
\end{lstlisting}

This script installs the required dependencies, downloads trafficserver (ATS) into \verb|/mnt|, makes trafficserver, and installs it into \verb|/opt/ts|.

\item

Change into the traffic server directory and start Apache Traffic Server in debug mode (stoppable by \verb|Ctrl-C|):

\begin{lstlisting} [language=bash] 
sudo /opt/ts/bin/traffic_server
\end{lstlisting}

To start or stop trafficserver as a daemon process:
\begin{lstlisting}[language=bash]
sudo /opt/ts/bin/trafficserver stop
sudo /opt/ts/bin/trafficserver start
\end{lstlisting}

\item Make some changes to the ATS proxy configurations file to set it up as a forward proxy:

\begin{lstlisting} [language=bash] 
cd /opt/ts/etc/trafficserver
\end{lstlisting}

Open the \verb|records.config| file and replace the following lines:

\begin{verbatim}
CONFIG proxy.config.reverse_proxy_enabled INT 1
CONFIG proxy.config.url_remap.remap_required INT 1
CONFIG proxy.config.http.cache.http INT 0
\end{verbatim}

With the following lines:

\begin{verbatim}
CONFIG proxy.config.reverse_proxy_enabled INT 0
CONFIG proxy.config.url_remap.remap_required INT 0
CONFIG proxy.config.http.cache.http INT 1
\end{verbatim}

\item Test the proxy with the following command:

\begin{lstlisting} [language=bash] 
curl —-proxy http://127.0.0.1:8080 -o /dev/null http://i.imgur.com/A2FUg9g.png
\end{lstlisting}

\item To view the logging, execute the following command:

\begin{lstlisting} [language=bash] 
/opt/ts/bin/traffic_logcat -f /opt/ts/var/log/trafficserver/squid.blog
\end{lstlisting}

Additionally, ATS can be started with the debug mode to view any logs with a given tag

\begin{lstlisting}[language=bash]
sudo /opt/ts/bin/traffic_server -T<DEBUG_TAG>
\end{lstlisting}


\end{enumerate}

\subsection{ATS with Cloud Caching}

\begin{enumerate}

\item
Download and install the AWS C++ SDK

\begin{lstlisting}[language=bash]
cd /mnt
git clone https://github.com/aws/aws-sdk-cpp.git
mkdir build-aws-sdk-cpp
cd build-aws-sdk-cpp
cmake ../aws-sdk-cpp -DBUILD_ONLY="aws-cpp-sdk-s2" -DCUSTOM_MEMORY_MANAGEMENT=0
make
sudo make install
\end{lstlisting}

\item
Point ATS cache Makefile to the local version of the headers:

Open \verb|/mnt/trafficserver/iocore/cache/Makefile.am| and update line 31 and 32 to reflect where your AWS C++ SDK headers reside. Most likely, they will be updated to the following:

\begin{verbatim}
-I/mnt/aws-sdk-cpp/aws-cpp-sdk-core/include \
-I/mnt/aws-sdk-cpp/aws-cpp-sdk-s3/include
\end{verbatim}

\item
Update the following lines in \verb|/opt/ts/etc/trafficserver/records.config|

\begin{verbatim}
CONFIG proxy.config.http.cache.cloud.enable INT 0
CONFIG proxy.config.http.cache.cloud.provider STRING NULL
\end{verbatim}

To the following to enable cloud caching with AWS:

\begin{verbatim}
CONFIG proxy.config.http.cache.cloud.enable INT 1
CONFIG proxy.config.http.cache.cloud.provider STRING "aws" 
\end{verbatim}

\item
Install the Redis command line interface from: \url{http://redis.io/download}

Set up an initial cluster with the following tutorial:

 \url{http://redis.io/topics/cluster-tutorial}

\item
Recompile ATS and run:

\begin{lstlisting}[language=bash]
cd /mnt/trafficserver
make && sudo make install
sudo /opt/ts/bin/traffic_server
\end{lstlisting}

\end{enumerate}

\subsection{Value-Based Caching Integration}

To run Flying Squid, run the following commands in the following order:

\begin{enumerate}

\item 

\begin{lstlisting} [language=bash] 
sudo /opt/ts/bin/trafficserver start
\end{lstlisting}

ATS will be running on port 8080.

\item 

Set Google chrome or Mozilla Firefox to point to the port that the TCP Proxy Client (Browser's Interfacing Layer) will run on. 

\item 

Start up the TCP Proxy Server (ATS's Interfacing Layer) in another terminal:

\begin{lstlisting} [language=bash] 
cd $(RABIN_FINGERPRINT_DIR)/Integration/tcpproxy
tcpproxy_server <local host ip> <tcpproxy_server port> <local host ip> 8080
\end{lstlisting}

\item

Start up the TCP Proxy Client (Browser's Interfacing Layer) in another terminal:

\begin{lstlisting} [language=bash] 
cd $(RABIN_FINGERPRINT_DIR)/Integration/tcpclient.
tcpproxy_client <local host ip> <tcpproxy_client port> <local host ip> <tcpproxy_server port>
\end{lstlisting}

\item

Open the browser that is pointing to the TCP Proxy Client and make request to a website, e.g. http://www.cs.tufts.edu.

\end{enumerate}

\subsection{Rabin Fingerprinting Demo}

\begin{enumerate}

\item 

Start up the Rabin Fingerprinting Server:

\begin{lstlisting} [language=bash] 
cd $(RABIN_FINGERPRINT_DIR)/Server
./rabinserver <Rabin Server port> ../Files/HTML/CSWebpage.html ../Files/HTML/CSWebpageWithoutHead.html
\end{lstlisting}

\item Start up the Rabin Fingerprinting Client in another terminal:

\begin{lstlisting} [language=bash] 
cd $(RABIN_FINGERPRINT_DIR)/Client
./rabinclient localhost <Rabin Server port> test.html
\end{lstlisting}

\end{enumerate}



\section{Better Integration: Google Native Client }

\paragraph{Idea}

The idea behind using Google Native Client \cite{GoogleNativeClient} is simply to eliminate the browser's interfacing layer. With Google Native Client, the browser can run C++ code in the browser. This way, the request does not have to pass through the TCP proxy interfacing layer and can go straight to ATS. Not only would this make Flying Squid more secure and portable, but it would also bring huge performance benefits.

In a web application, Google Native Client is embedded within the HTML through the use of an <embed> tag. The JavaScript and Google Native Client modules talk to each other via bidirectional, asynchronous messages.

\paragraph{Setting up Native Client}

Before Native Client can be set up, Python 2.7 and Make need to be available on the machine. Once these executables are available, the Native Client executable can be downloaded at the following website: https://developer.chrome.com/native-client/sdk/download. A Hello World Tutorial is also available at https://developer.chrome.com/native-client/devguide/tutorial/tutorial-part1.

For our purposes, however, perform the following steps to get up to speed:

\begin{enumerate} 
\item Clone the clientInterfacing directory. The command is 

\begin{lstlisting} [language=bash] 
git clone https://github.com/FlyingSquid/clientInterfacing.git
\end{lstlisting}

\item  Make sure your Google Chrome browser is version 49 or higher.

\item Change into the directory \verb|nacl_sdk/pepper_49/getting_started/part1|.

\item Execute \verb|make|.

\item Change into the directory \verb|nacl_sdk/pepper_49/getting_started|.

\item Execute \verb|make serve|.

\item Use the Google Chrome browser to access \verb|http://localhost:5103|. 

\end{enumerate}

\paragraph{Debugging}

The application can also be opened on MAC OSX in the terminal via the following commands:

\begin{lstlisting} [language=bash]
mkdir ~/Desktop/testtest

cd /Applications/Google Chrome.app/Contents/MacOS

./Google\ Chrome http://localhost:5103/part1 --enable-logging --v=1 --user-data-dir=~/Desktop/testtest
\end{lstlisting}

This will enable logging to appear, making it easier for future developers to debug the Native Client module.


\paragraph{Challenges}

Unfortunately, there were a few roadblocks that prevented Google Native Client from being a viable option for Flying squid at the moment. The roadblocks are listed below and are left for future developers to tackle.

\begin{enumerate}

\item Google Native Client cannot be compiled with g++ and the -std=c++0x flag, both of which are required to compile the Rabin Client and Rabin Server correctly. 

\item Google Native Client cannot access the local filesystem of the machine that it is running on. It can only simulate a File I/O API using a ocal secure data store. This File I/O API is described at  \url{https://developer.chrome.com/native-client/devguide/}coding/file-io. Further documentation of implementations of standard POSIX I/O functions such as fopen, fseek, fread, fwirte, and fclose are described at \url{https://developer.chrome.com/native-client/devguide/coding/nacl_io}.
\end{enumerate}

\end{appendices}